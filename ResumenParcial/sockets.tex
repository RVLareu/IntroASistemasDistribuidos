
\subsection{TCP}

\begin{lstlisting}[caption= Cliente TCP]
from socket import *
serverName = "servername"
serverPort = 12000
clientSocket = socket(AF_INET, SOCK_STREAM)
clientSocket.connect((serverName, serverPort))
sentence = raw_input("Input lowercase sentence:")
clientSocket.send(sentence.encode())
modifiedSentence = clientSocket.recv(1024)
print("From Server: ", modifiedSentence.decode())
clientSocket.close()

\end{lstlisting}

Se crea el cliente, \textbf{AF\_INET} indica que la red usará IPv4, el segundo parámetro indica que es TCP socket. El SO define el puerto a usar. Con el método \textbf{connect} se inicia la conexion entre el cliente y el servidor, pasando por parámetro la direccion y puerto del servidor. Luego de esta línea, ya se estableció el three way handshake. Con la funcion send, el cliente pone la tira de bytes en la conexion TCP y luego espera la respuesta. Finalmente se corta la conexión TCP


\begin{lstlisting}[caption= Servidor TCP]
from socket import *
serverPort = 12000
serverSocket = socket(AF_INET, SOCK_STREAM)
serverSocket.bind(("", serverPort))
serverSocket.listen(1)
print("The server is ready to receive")
while True:
    connectionSocket, addr = serverSocket.accept()
    sentence = connectionSocket.recv(1024).decode()
    capitalizedSentence = sentence.upper()
    connectionSocket.send(capitalizedSentence.encode())
    connectionSocket.close()
\end{lstlisting}

Se crea el socket y se asocia el puerto con el socket recientemente creado (bind). Este será el socket de bienvenida, el cual queda a la escucha de nuevos clientes (listen). El parámetro indica la máxima cantidad de conexiones encoladas. Cuando llega un cliente, invoca el método accept, el cual crea un nuevo socket dedicado a ese cliente en particular. Se realiza el handshake y se establece la conexion. A partir de ahora pueden enviarse bytes sobre la conexion establecida. Finalmente se cierra la conexion, pero serverSocket sigue abierto, a la espera de nuevos clientes

\subsection{UDP}

\begin{lstlisting}[caption = Cliente UDP]
from socket import *
serverName = "hostname"
serverPort = 12000
clientSocket = socket(AF_INET, SOCK_DGRAM)
message = raw_input("Input lowercase sentence:")
clientSocket.sendto(message.encode(),(serverName, serverPort))
modifiedMessage, serverAddress = clientSocket.recvfrom(2048)
print(modifiedMessage.decode())
clientSocket.close()
\end{lstlisting}

El parámetro \textbf{SOCK\_DGRAM} indica que será un socket UDP. Con el método \textit{sendto} se envía el mensaje (en tipo bytes) indicando la direccion y puerto del servidor. Luego espera la respuesta del servidor. Puede extraerse la dirección del servidor al recibir el mensaje.

\begin{lstlisting}[caption = Servidor UDP]
from socket import *
serverPort = 12000
serverSocket = socket(AF_INET, SOCK_DGRAM)
serverSocket.bind(("", serverPort))
print(” The server is ready to receive...”)
while True:
    message, clientAddress = serverSocket.recvfrom(2048)
    modifiedMessage = message.decode().upper()
    serverSocket.sendto(modifiedMessage.encode(), clientAddress)
\end{lstlisting}

Se asigna el puerto al socket con \textit{bind}. Cuando cualquiera envie un paquete a la direccion Ip y a ese puerto, llegará al socket. El loop permite recibir y procesar paquetes de varios clientes indefinidamente. Allí espera el arrivo de paquetes. La variable \textit{clientAddress} contiene la direccion IP del cliente y el puerto. En \textit{sendto} se especifica la direccion del cliente destino (dirección y puerto)